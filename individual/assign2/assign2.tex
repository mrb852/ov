\documentclass[12pt,a4paper]{article}

\setlength{\textwidth}{165mm}
\setlength{\textheight}{240mm}
\setlength{\parindent}{0mm} % S{\aa} meget rykkes ind efter afsnit
\setlength{\parskip}{\parsep}
\setlength{\headheight}{0mm}
\setlength{\headsep}{0mm}
\setlength{\hoffset}{-2.5mm}
\setlength{\voffset}{0mm}
\setlength{\footskip}{15mm}
\setlength{\oddsidemargin}{0mm}
\setlength{\topmargin}{0mm}
\setlength{\evensidemargin}{0mm}

\usepackage{amsmath} % flere matematikkommandoer
\makeatletter
\usepackage[all]{xy}
\usepackage{graphicx}    % For grafik (billederfiler)
\usepackage[T1]{fontenc} % For at blande \textsc{} med \textbf{}
\usepackage[utf8]{inputenc}
\usepackage{amsfonts,amsmath,amssymb}
\usepackage{eucal}
\usepackage[danish]{babel}
\usepackage{enumerate}  
\usepackage{hyperref}
\usepackage{url}
\renewcommand*\env@matrix[1][*\c@MaxMatrixCols c]{%
  \hskip -\arraycolsep
  \let\@ifnextchar\new@ifnextchar
  \array{#1}}
\usepackage{mathptmx}

\usepackage{multirow}
\usepackage[dvipsnames,usenames]{color}
\usepackage{tabularx,colortbl,xcolor}
\definecolor{KU-red}{RGB}{144,26,30} 

\DeclareSymbolFont{usualmathcal}{OMS}{cmsy}{m}{n}
\DeclareSymbolFontAlphabet{\mathcal}{usualmathcal}


\DeclareSymbolFont{letters}{OML}{txmi}{m}{it}

\DeclareMathSymbol{\alpha}{\mathord}{letters}{"0B}
\DeclareMathSymbol{\beta}{\mathord}{letters}{"0C}
\DeclareMathSymbol{\gamma}{\mathord}{letters}{"0D}
\DeclareMathSymbol{\delta}{\mathord}{letters}{"0E}
\DeclareMathSymbol{\epsilon}{\mathord}{letters}{"0F}
\DeclareMathSymbol{\zeta}{\mathord}{letters}{"10}
\DeclareMathSymbol{\eta}{\mathord}{letters}{"11}
\DeclareMathSymbol{\theta}{\mathord}{letters}{"12}
\DeclareMathSymbol{\iota}{\mathord}{letters}{"13}
\DeclareMathSymbol{\kappa}{\mathord}{letters}{"14}
\DeclareMathSymbol{\lambda}{\mathord}{letters}{"15}
\DeclareMathSymbol{\mu}{\mathord}{letters}{"16}
\DeclareMathSymbol{\nu}{\mathord}{letters}{"17}
\DeclareMathSymbol{\xi}{\mathord}{letters}{"18}
\DeclareMathSymbol{\pi}{\mathord}{letters}{"19}
\DeclareMathSymbol{\rho}{\mathord}{letters}{"1A}
\DeclareMathSymbol{\sigma}{\mathord}{letters}{"1B}
\DeclareMathSymbol{\tau}{\mathord}{letters}{"1C}
\DeclareMathSymbol{\upsilon}{\mathord}{letters}{"1D}
\DeclareMathSymbol{\phi}{\mathord}{letters}{"1E}
\DeclareMathSymbol{\chi}{\mathord}{letters}{"1F}
\DeclareMathSymbol{\psi}{\mathord}{letters}{"20}
\DeclareMathSymbol{\omega}{\mathord}{letters}{"21}
\DeclareMathSymbol{\varepsilon}{\mathord}{letters}{"22}
\DeclareMathSymbol{\vartheta}{\mathord}{letters}{"23}
\DeclareMathSymbol{\varpi}{\mathord}{letters}{"24}
\DeclareMathSymbol{\varrho}{\mathord}{letters}{"25}
\DeclareMathSymbol{\varsigma}{\mathord}{letters}{"26}
\DeclareMathSymbol{\varphi}{\mathord}{letters}{"27}

\DeclareMathSymbol{\Gamma}{\mathord}{letters}{"00}
\DeclareMathSymbol{\Delta}{\mathord}{letters}{"01}
\DeclareMathSymbol{\Theta}{\mathord}{letters}{"02}
\DeclareMathSymbol{\Lambda}{\mathord}{letters}{"03}
\DeclareMathSymbol{\Xi}{\mathord}{letters}{"04}
\DeclareMathSymbol{\Pi}{\mathord}{letters}{"05}
\DeclareMathSymbol{\Sigma}{\mathord}{letters}{"06}
\DeclareMathSymbol{\Upsilon}{\mathord}{letters}{"07}
\DeclareMathSymbol{\Phi}{\mathord}{letters}{"08}
\DeclareMathSymbol{\Psi}{\mathord}{letters}{"09}
\DeclareMathSymbol{\Omega}{\mathord}{letters}{"0A}

\DeclareMathSymbol{\upGamma}{\mathalpha}{operators}{"00}
\DeclareMathSymbol{\upDelta}{\mathalpha}{operators}{"01}
\DeclareMathSymbol{\upTheta}{\mathalpha}{operators}{"02}
\DeclareMathSymbol{\upLambda}{\mathalpha}{operators}{"03}
\DeclareMathSymbol{\upXi}{\mathalpha}{operators}{"04}
\DeclareMathSymbol{\upPi}{\mathalpha}{operators}{"05}
\DeclareMathSymbol{\upSigma}{\mathalpha}{operators}{"06}
\DeclareMathSymbol{\upUpsilon}{\mathalpha}{operators}{"07}
\DeclareMathSymbol{\upPhi}{\mathalpha}{operators}{"08}
\DeclareMathSymbol{\upPsi}{\mathalpha}{operators}{"09}
\DeclareMathSymbol{\upOmega}{\mathalpha}{operators}{"0A}



\newcommand{\hhemail}[1]{\textsf{#1}}
\newcommand{\hhurl}[1]{{\color{blue}\url{#1}}}

\begin{document}

\newpage

Christian Enevoldsen\\

\textbf{Exercise 1}

\textbf{a}\\
Det ville printe x: 3 og y: 1, fordi de ikke bliver ændret når man bruger call by value.\\


\textbf{b}\\
Den printer flg. fordi a og c peger til samme adresse.
r: 19
x: 8, y: 3\\

\textbf{c}\\
Den ville printe flg. fordi at y og x bliver ændret i funktionen. x bliver ændret 2 gange, men den sidste reference er ved c.
r: 13
x: 7
y: 2\\

\textbf{Exercise 2}

\textbf{a}\\
f(3) ville printe x: 4, da den tjekker om den lokale x er tre, hvilket det er og kalder en funktion g som bare printer den globale x.\\

f(5) ville printe printe 4 af samme grund.

\textbf{b}\\
f(3) ville printe x: 3 da den ville overskrive den første erklærede instans af x, hvilket er den globale. Derefter er x = 3 hvilket forårsager at g bliver kaldt og den printer så x: 3

f(5) ville printe 7, fordi den globale x altid bliver overskrevet af nye erklæringer.\\

\textbf{exercise 3}

\textbf{a}\\
fun char main() =
  let x = read(char) in
    if x == 'c' then x else 1

I eksemplet ovenfor vil interpreteren ikke opdage fejlen før if statementet og derfor vil den først printe en fejl når x != 'c'.

\textbf{b} \\

Scan: \\
$\forall a . \forall b. ((a * b) \rightarrow b * [a]) \rightarrow [b] $

1. Udregn typen af et element fra $a: t$\\
2. udregn typen af expr $e: te$ \\
3. Tjek om typerne matcher i $ftable$ \\	
4. IF $argc$ (arg count)$!= 2$ THEN error ELSE good\\
5. IF $typeof(argv[1]) == t $ AND $ typeof(argv[2]) == te$ THEN good ELSE error\\
6. IF$ typeof(retVal) == t $ THEN good ELSE error\\

Filter: \\
$\forall a (a \rightarrow b * [a]) \rightarrow [a], b = bool $


1. Udregn typen af et element fra $a: t$\\
2. Tjek om typen matcher i $ftable$ \\	
3. IF $argc$ (arg count)$\ne 1$ THEN error ELSE good\\
4. IF $typeof(argv[1]) == bool$ THEN good ELSE error\\
5. IF $ typeof(retVal) == bool $ THEN good ELSE error\\

\end{document}

